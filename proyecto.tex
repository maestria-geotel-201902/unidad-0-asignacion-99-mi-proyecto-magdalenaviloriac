\documentclass[11pt,]{article}
\usepackage[left=1in,top=1in,right=1in,bottom=1in]{geometry}
\newcommand*{\authorfont}{\fontfamily{phv}\selectfont}
\usepackage[]{mathpazo}


  \usepackage[T1]{fontenc}
  \usepackage[utf8]{inputenc}



\usepackage{abstract}
\renewcommand{\abstractname}{}    % clear the title
\renewcommand{\absnamepos}{empty} % originally center

\renewenvironment{abstract}
 {{%
    \setlength{\leftmargin}{0mm}
    \setlength{\rightmargin}{\leftmargin}%
  }%
  \relax}
 {\endlist}

\makeatletter
\def\@maketitle{%
  \newpage
%  \null
%  \vskip 2em%
%  \begin{center}%
  \let \footnote \thanks
    {\fontsize{18}{20}\selectfont\raggedright  \setlength{\parindent}{0pt} \@title \par}%
}
%\fi
\makeatother




\setcounter{secnumdepth}{3}

\usepackage{color}
\usepackage{fancyvrb}
\newcommand{\VerbBar}{|}
\newcommand{\VERB}{\Verb[commandchars=\\\{\}]}
\DefineVerbatimEnvironment{Highlighting}{Verbatim}{commandchars=\\\{\}}
% Add ',fontsize=\small' for more characters per line
\usepackage{framed}
\definecolor{shadecolor}{RGB}{248,248,248}
\newenvironment{Shaded}{\begin{snugshade}}{\end{snugshade}}
\newcommand{\KeywordTok}[1]{\textcolor[rgb]{0.13,0.29,0.53}{\textbf{#1}}}
\newcommand{\DataTypeTok}[1]{\textcolor[rgb]{0.13,0.29,0.53}{#1}}
\newcommand{\DecValTok}[1]{\textcolor[rgb]{0.00,0.00,0.81}{#1}}
\newcommand{\BaseNTok}[1]{\textcolor[rgb]{0.00,0.00,0.81}{#1}}
\newcommand{\FloatTok}[1]{\textcolor[rgb]{0.00,0.00,0.81}{#1}}
\newcommand{\ConstantTok}[1]{\textcolor[rgb]{0.00,0.00,0.00}{#1}}
\newcommand{\CharTok}[1]{\textcolor[rgb]{0.31,0.60,0.02}{#1}}
\newcommand{\SpecialCharTok}[1]{\textcolor[rgb]{0.00,0.00,0.00}{#1}}
\newcommand{\StringTok}[1]{\textcolor[rgb]{0.31,0.60,0.02}{#1}}
\newcommand{\VerbatimStringTok}[1]{\textcolor[rgb]{0.31,0.60,0.02}{#1}}
\newcommand{\SpecialStringTok}[1]{\textcolor[rgb]{0.31,0.60,0.02}{#1}}
\newcommand{\ImportTok}[1]{#1}
\newcommand{\CommentTok}[1]{\textcolor[rgb]{0.56,0.35,0.01}{\textit{#1}}}
\newcommand{\DocumentationTok}[1]{\textcolor[rgb]{0.56,0.35,0.01}{\textbf{\textit{#1}}}}
\newcommand{\AnnotationTok}[1]{\textcolor[rgb]{0.56,0.35,0.01}{\textbf{\textit{#1}}}}
\newcommand{\CommentVarTok}[1]{\textcolor[rgb]{0.56,0.35,0.01}{\textbf{\textit{#1}}}}
\newcommand{\OtherTok}[1]{\textcolor[rgb]{0.56,0.35,0.01}{#1}}
\newcommand{\FunctionTok}[1]{\textcolor[rgb]{0.00,0.00,0.00}{#1}}
\newcommand{\VariableTok}[1]{\textcolor[rgb]{0.00,0.00,0.00}{#1}}
\newcommand{\ControlFlowTok}[1]{\textcolor[rgb]{0.13,0.29,0.53}{\textbf{#1}}}
\newcommand{\OperatorTok}[1]{\textcolor[rgb]{0.81,0.36,0.00}{\textbf{#1}}}
\newcommand{\BuiltInTok}[1]{#1}
\newcommand{\ExtensionTok}[1]{#1}
\newcommand{\PreprocessorTok}[1]{\textcolor[rgb]{0.56,0.35,0.01}{\textit{#1}}}
\newcommand{\AttributeTok}[1]{\textcolor[rgb]{0.77,0.63,0.00}{#1}}
\newcommand{\RegionMarkerTok}[1]{#1}
\newcommand{\InformationTok}[1]{\textcolor[rgb]{0.56,0.35,0.01}{\textbf{\textit{#1}}}}
\newcommand{\WarningTok}[1]{\textcolor[rgb]{0.56,0.35,0.01}{\textbf{\textit{#1}}}}
\newcommand{\AlertTok}[1]{\textcolor[rgb]{0.94,0.16,0.16}{#1}}
\newcommand{\ErrorTok}[1]{\textcolor[rgb]{0.64,0.00,0.00}{\textbf{#1}}}
\newcommand{\NormalTok}[1]{#1}

\usepackage{graphicx,grffile}
\makeatletter
\def\maxwidth{\ifdim\Gin@nat@width>\linewidth\linewidth\else\Gin@nat@width\fi}
\def\maxheight{\ifdim\Gin@nat@height>\textheight\textheight\else\Gin@nat@height\fi}
\makeatother
% Scale images if necessary, so that they will not overflow the page
% margins by default, and it is still possible to overwrite the defaults
% using explicit options in \includegraphics[width, height, ...]{}
\setkeys{Gin}{width=\maxwidth,height=\maxheight,keepaspectratio}

\title{\textbar{} Mi proyecto \textbar{}Datos puntuales y Geoestadistica
\textbar{} Subtítulo \textbar{}Proyecto final de Analisis Espacial
\textbar{} Profesor: \textbar{}Jose Martinez batlle  }



\author{\Large Maria Magdalena Viloria Y Adalberto Guerrero\vspace{0.05in} \newline\normalsize\emph{Estudiante, Universidad Autónoma de Santo Domingo (UASD)}  }


\date{}

\usepackage{titlesec}

\titleformat*{\section}{\normalsize\bfseries}
\titleformat*{\subsection}{\normalsize\itshape}
\titleformat*{\subsubsection}{\normalsize\itshape}
\titleformat*{\paragraph}{\normalsize\itshape}
\titleformat*{\subparagraph}{\normalsize\itshape}

\titlespacing{\section}
{0pt}{36pt}{0pt}
\titlespacing{\subsection}
{0pt}{36pt}{0pt}
\titlespacing{\subsubsection}
{0pt}{36pt}{0pt}





\newtheorem{hypothesis}{Hypothesis}
\usepackage{setspace}

\makeatletter
\@ifpackageloaded{hyperref}{}{%
\ifxetex
  \PassOptionsToPackage{hyphens}{url}\usepackage[setpagesize=false, % page size defined by xetex
              unicode=false, % unicode breaks when used with xetex
              xetex]{hyperref}
\else
  \PassOptionsToPackage{hyphens}{url}\usepackage[unicode=true]{hyperref}
\fi
}

\@ifpackageloaded{color}{
    \PassOptionsToPackage{usenames,dvipsnames}{color}
}{%
    \usepackage[usenames,dvipsnames]{color}
}
\makeatother
\hypersetup{breaklinks=true,
            bookmarks=true,
            pdfauthor={Maria Magdalena Viloria Y Adalberto Guerrero (Estudiante, Universidad Autónoma de Santo Domingo (UASD))},
             pdfkeywords = {geoestadística, Datos puntuales},  
            pdftitle={\textbar{} Mi proyecto \textbar{}Datos puntuales y Geoestadistica
\textbar{} Subtítulo \textbar{}Proyecto final de Analisis Espacial
\textbar{} Profesor: \textbar{}Jose Martinez batlle},
            colorlinks=true,
            citecolor=blue,
            urlcolor=blue,
            linkcolor=magenta,
            pdfborder={0 0 0}}
\urlstyle{same}  % don't use monospace font for urls

% set default figure placement to htbp
\makeatletter
\def\fps@figure{htbp}
\makeatother

\usepackage{pdflscape} \newcommand{\blandscape}{\begin{landscape}}
\newcommand{\elandscape}{\end{landscape}}


% add tightlist ----------
\providecommand{\tightlist}{%
\setlength{\itemsep}{0pt}\setlength{\parskip}{0pt}}

\begin{document}
	
% \pagenumbering{arabic}% resets `page` counter to 1 
%
% \maketitle

{% \usefont{T1}{pnc}{m}{n}
\setlength{\parindent}{0pt}
\thispagestyle{plain}
{\fontsize{18}{20}\selectfont\raggedright 
\maketitle  % title \par  

}

{
   \vskip 13.5pt\relax \normalsize\fontsize{11}{12} 
\textbf{\authorfont Maria Magdalena Viloria Y Adalberto Guerrero} \hskip 15pt \emph{\small Estudiante, Universidad Autónoma de Santo Domingo (UASD)}   

}

}








\begin{abstract}

    \hbox{\vrule height .2pt width 39.14pc}

    \vskip 8.5pt % \small 

\noindent se presenta un conjunto de procesos y datos de precipitación, para ser
usados en el paquete de R donde este permitirá su manipulación a través
de sus distintas funcionalidades para obtener Datos puntuales y
Geoestadistica para el año 1998 de las precipitaciones


\vskip 8.5pt \noindent \emph{Keywords}: geoestadística, Datos puntuales \par

    \hbox{\vrule height .2pt width 39.14pc}



\end{abstract}


\vskip 6.5pt


\noindent  \section{Introducción}\label{introducciuxf3n}

Hemos tomado la precipitaciones del año 1998 para realizar una series de
analisis geoestadistico donde cargaremos los datos de precipitaciones de
la ONAMET en los cuales obtuvimos valores en los lugares no muestrados,
se realizo prediciones basado en un año 1998, utilizando los elementos
del variograma, el efecto pepita ademas generamos el variograma
muestral, obtubimos el efecto proximo mediante el metodo Kriging
ordinario, se detecto la autocorrelación espacial, es decir, relación
entre observaciones debido a la estructrura espacial determinamos Si
existe autocorrelación para que ésta pueda modelizarse adecuadamente,
mediante un variograma modelo obtuvimos los datos estadisticos y la
distribución que estos siguen.

las técnicas discutidas en este trabajo son realizada mediante los
paquetes estadísticos de R ya que cuenta con una enorme variedad de
pruebas estadísticas y representaciones graficas.

\section{Metodología}\label{metodologuxeda}

\ldots

\section{Datos puntuales,
geoestadística.}\label{datos-puntuales-geoestaduxedstica.}

\subsection{Geoestadística}\label{geoestaduxedstica}

La Geoestadística es la rama de la estadística que se encarga del
análisis de fenómenos espaciales que exhiben un comportamiento
estructural, en este caso se centrado en el análisis de la variable
precipitación a partir de cuatro métodos de interpolación que permitan
obtener una delimitación más acorde del área de estudio \#\#\#
Definición

La geoestadística se ocupa en modelar, predecir y simular fenómenos
espacialmente continuos. En realidad, lo que aborda es la obtención de
valores en lugares no muestreados. Dado que es imposible tomar muestras
en todas partes, la geoestadística asiste en la predicción espacialmente
continua del valor de una variable.

\subsubsection{Kriging}\label{kriging}

Considera un fenómeno \emph{Z(s)}, muestreado en \emph{si}, con
\emph{i=1,2,3,\ldots{},n}, pero nos interesa su valor en todas las
localidades, \emph{s0}; debemos predecir \emph{s0}, lo cual realizaremos
de manera habitual con el método del krigeaje (krigeado o
\textbf{kriging}), que consiste en una predicción Gaussiana.

Existen varias modalidades de krigeaje según los distintos supuestos
(todas asumen que la variación espacial es modelizable mediante el
variograma), siendo las más comunes las siguientes:

\begin{itemize}
\item
  \textbf{Kriging ordinario}. Asume \textbf{media constante y
  desconocida} en el entorno próximo de la observación \emph{si}.
  \textbf{No se asume que la media local tenga el mismo valor que la
  media poblacional}, es decir, no hay requisito de estacionariedad de
  primer momento. Dada las pocas asunciones que requiere en su
  implementación, y por su aproximación más conservadora, es muy
  empleado en varias disciplinas.
\item
  \textbf{Kriging simple}. \textbf{Asume estacionariedad de primer
  momento} (media constante) en el dominio analizado, pero la media
  poblacional es conocida.
\item
  \textbf{Kriging universal}. Asume una \textbf{tendencia general
  modelizada por polinomio}, como por ejemplo, un modelo lineal.
\end{itemize}

\subsubsection{Semivarianza y
variograma}\label{semivarianza-y-variograma}

El kriging utiliza un modelo para ponderar las observaciones cercanas a
\emph{s0} denominado variograma o semivariograma. El variograma es el
gráfico de representación de estimaciones de la semivarianza γ; esta
última mide el grado de dependencia espacial entre muestras. En la
medida en que aumenta la distancia entre pares de observaciones, se
asume que la semivarianza aumenta igualmente. Observaciones cercanas
obtendrán semivarianzas pequeñas, puesto que si existe autocorrelación
espacial, observaciones cercanas serán muy parecidas.

Las estimaciones de la semivarianza se obtienen a partir de la fórmula:

\emph{Tomado de ({\textbf{???}})}

Donde \emph{hi={[}hi,0,hi,1{]}} es el intervalo de distancia para
\emph{i} y \emph{N(hi)} es el número de pares de muestras disponibles
para \emph{i} en el intervalo \emph{hi}, \emph{z(si)} es el valor de la
variable en \emph{i}, y \emph{z(si)+h'} el valor de la variable a la
distancia \emph{h'} desde \emph{i}.

En realidad, los programas de computo de la semivarianza utilizan un
diseño geométrico de búsqueda basado en \emph{lags} que se fijan de
incrementalmente de acuerdo al tamaño de intervalo. Aunque la anchura de
búsqueda es fija, los \emph{lags} pueden ser múltiples, lo que complica
significativamente los cálculos.

\subsection{Paquetes}\label{paquetes}

\begin{Shaded}
\begin{Highlighting}[]
\KeywordTok{library}\NormalTok{(gstat)}
\KeywordTok{show.vgms}\NormalTok{()}
\end{Highlighting}
\end{Shaded}

\includegraphics{proyecto_files/figure-latex/variogramas-modelo-1.pdf}

\begin{Shaded}
\begin{Highlighting}[]
\KeywordTok{vgm}\NormalTok{()}
\end{Highlighting}
\end{Shaded}

\begin{verbatim}
##    short                                      long
## 1    Nug                              Nug (nugget)
## 2    Exp                         Exp (exponential)
## 3    Sph                           Sph (spherical)
## 4    Gau                            Gau (gaussian)
## 5    Exc        Exclass (Exponential class/stable)
## 6    Mat                              Mat (Matern)
## 7    Ste Mat (Matern, M. Stein's parameterization)
## 8    Cir                            Cir (circular)
## 9    Lin                              Lin (linear)
## 10   Bes                              Bes (bessel)
## 11   Pen                      Pen (pentaspherical)
## 12   Per                            Per (periodic)
## 13   Wav                                Wav (wave)
## 14   Hol                                Hol (hole)
## 15   Log                         Log (logarithmic)
## 16   Pow                               Pow (power)
## 17   Spl                              Spl (spline)
## 18   Leg                            Leg (Legendre)
## 19   Err                   Err (Measurement error)
## 20   Int                           Int (Intercept)
\end{verbatim}

Los elementos de un variograma son los siguientes (según
({\textbf{???}})):

\begin{itemize}
\item
  Efecto ``pepita'' (\emph{nugget}): corresponde a la variabilidad
  aleatoria inherente {[}variabilidad inframuestra, imposible de
  detectarse con la espaciación elegida en el diseño muestral{]}.
\item
  Rango (\emph{range}), rango de influencia, alcance: corresponde a la
  distancia a la cual las muestras se tornan independientes, es la
  distancia donde se alcanza la meseta.
\item
  Meseta (\emph{psill}): correspondiente a la varianza estructural.
\end{itemize}

\subsubsection{¿Cómo interpretar un
variograma?}\label{cuxf3mo-interpretar-un-variograma}

La principal tarea al generar un variograma consiste en detectar
autocorrelación espacial, es decir, relación entre observaciones debido
a la estructrura espacial. Si existe autocorrelación, ésta podrá
modelizarse adecuadamente mediante un variograma modelo.

El variograma de una variable autocorrelacionada espacialmente muestra
un incremento gradual de la semivarianza hasta que se alcanza la meseta
en el rango (concretamente, al alcanzar el 95\% de la semivarianza
total). Esto implica que la semivarianza inicia en cero o cercana a éste
para el intervalo de distancia cero.

En el caso de una variable no autocorrelacionada, ya sea porque tiene un
comportamiento aleatorio o porque la distribución espacial de las
observaciones es demasiado gruesa como para detectar la autocorrelación
de escalas más detalladas (variabilidad inframuestra), en la distancia
cero el variograma intercepta el eje de la semivarianza en valores
significativamente elevados.

Nótese por ejemplo el caso del variograma LMVE\_02, donde la
semivarianza aumenta gradualmente hasta alcanzar el rango, que ocurre
entre 5 y 10 m. Por el contrario, EPCA\_02 y LIv\_1 no presentan un
patrón autocorrelacionado, puesto que la semivarianza no aumenta
gradualmente.

\subsection{Estudio de caso: precipitación de República Dominicana
mediante kriging
ordinario}\label{estudio-de-caso-precipitaciuxf3n-de-repuxfablica-dominicana-mediante-kriging-ordinario}

En esta introducción realizaremos un \textbf{kriging ordinario} y un
\textbf{kriging universal} utilizando los observatorios de precipitación
de República Dominicana colectados por ONAMET.

\subsubsection{Datos fuente}\label{datos-fuente}

Tomaremos como ejemplo la precipitación del año 1998. Primero carguemos
los observatorios y las provincias:

\begin{Shaded}
\begin{Highlighting}[]
\KeywordTok{library}\NormalTok{(sf)}
\NormalTok{rutapre <-}\StringTok{ 'material-de-apoyo-master/data/onamet_prec_anual_sf.gpkg'}
\NormalTok{rutadiv <-}\StringTok{ 'material-de-apoyo-master/data/divisionRD.gpkg'}
\KeywordTok{st_layers}\NormalTok{(rutapre)}
\end{Highlighting}
\end{Shaded}

\begin{verbatim}
## Driver: GPKG 
## Available layers:
##             layer_name geometry_type features fields
## 1 onamet_prec_anual_sf         Point       25     37
\end{verbatim}

\begin{Shaded}
\begin{Highlighting}[]
\NormalTok{pre <-}\StringTok{ }\KeywordTok{st_read}\NormalTok{(rutapre)}
\end{Highlighting}
\end{Shaded}

\begin{verbatim}
## Reading layer `onamet_prec_anual_sf' from data source `/home/magda/unidad-0-asignacion-99-mi-proyecto-magdalenaviloriac/material-de-apoyo-master/data/onamet_prec_anual_sf.gpkg' using driver `GPKG'
## Simple feature collection with 25 features and 37 fields
## geometry type:  POINT
## dimension:      XY
## bbox:           xmin: -71.7 ymin: 18.067 xmax: -68.367 ymax: 19.85
## epsg (SRID):    4326
## proj4string:    +proj=longlat +datum=WGS84 +no_defs
\end{verbatim}

\begin{Shaded}
\begin{Highlighting}[]
\NormalTok{pre}
\end{Highlighting}
\end{Shaded}

\begin{verbatim}
## Simple feature collection with 25 features and 37 fields
## geometry type:  POINT
## dimension:      XY
## bbox:           xmin: -71.7 ymin: 18.067 xmax: -68.367 ymax: 19.85
## epsg (SRID):    4326
## proj4string:    +proj=longlat +datum=WGS84 +no_defs
## First 10 features:
##            Estación  a1979  a1980  a1981  a1982  a1983  a1984  a1985
## 1          Barahona 1740.0 1053.6 1435.3  815.3 1183.0  584.1  997.8
## 2         Bayaguana 2794.3 1761.5 2412.4 1758.6 1857.1 1645.6 1928.3
## 3           Cabrera 2035.0 1276.8     NA 2136.9 1703.8 1888.7 1557.1
## 4         Constanza 1652.1 1166.9 1343.3  921.2  828.4     NA  892.8
## 5  Gaspar Hernández     NA 1443.8 2174.9 1844.1 1688.8 2208.8 1895.5
## 6       Hondo Valle 1823.6 1778.2 2203.7 1709.9 1841.3 1796.6 1309.5
## 7            Jimaní 1060.7  639.1  960.2  507.5  610.7  641.5  689.6
## 8          La Unión 1781.5 1630.6 2304.4 1413.1 1288.4 1499.4 1157.1
## 9           La Vega 1833.5 1304.3 1993.7 1483.2 1353.9 1550.1 1084.9
## 10     Las Américas 1958.4  958.7 1513.4  787.4  975.5  954.9 1398.2
##     a1986  a1987  a1988  a1989   a1990  a1991  a1992  a1993  a1994   a1995
## 1  1080.0 1423.9  704.7 1011.6 1075.20  983.1 1112.5  968.5 1622.4  956.00
## 2  2182.2 2273.5 1813.2 1730.6 1823.40 1850.3 1765.7 1606.2 1892.8 1360.10
## 3  1597.0 2059.7     NA 1176.9 1183.40  957.6     NA     NA     NA      NA
## 4   715.8  786.9  837.7  671.5  875.35     NA  858.6  858.6  900.7  839.40
## 5  2874.7 2360.8 1426.3 1214.2 1530.70     NA 1257.5 1345.3 1824.9 1665.45
## 6  1589.7 1778.8 1766.5 1722.8 1596.10 1088.4 1731.0 1887.0 1772.0 1288.30
## 7   802.4  648.9  521.0  680.7  880.00  311.6  809.2  472.9  840.2  909.00
## 8  1313.1 1786.5 1888.8 1222.8 1808.00 1250.4 1555.2 1484.8 1035.9  877.70
## 9  1767.1 1663.2 1934.9 1192.4 1664.40 1146.4 1565.6 1855.4 1455.7 1175.40
## 10 1419.0 1866.4 1620.5 1151.7      NA  997.0     NA     NA     NA 1017.50
##      a1996   a1997  a1998  a1999  a2000  a2001  a2002   a2003  a2004
## 1   965.65  662.60  684.6  662.7  600.0  600.0  997.6  942.60  972.6
## 2  1867.70 1618.60 2156.6 1712.5 1868.5 1796.1 1658.0 2117.30 1554.2
## 3       NA      NA     NA     NA 1538.6 1852.9  946.9 1810.95 2053.3
## 4  1167.30  525.10 1492.7 1077.8  951.3  787.1  959.2 1084.10  985.9
## 5  2656.80  984.80 2147.9 1791.9 1716.9 2178.8 1093.4 2058.50 1906.8
## 6  1447.90  912.65 1813.9 1762.2 2285.9 1604.3 1477.4 1628.10 1617.7
## 7   816.20  358.20  824.1 1037.0  833.9  488.4  510.1  656.70  866.9
## 8  1980.50  554.20 1744.1 1314.3 1148.5 1360.5  972.1 1802.00 2550.1
## 9  1772.50 1018.80 1549.6 1817.9 1368.6 1522.0 1200.7 2290.60 1825.7
## 10 1019.60  651.20 1218.6 1125.9  809.7  747.6  933.4 1083.60 1338.9
##      a2005   a2006   a2007   a2008  a2009  a2010  a2011  a2012  a2013
## 1  1274.60 1118.40 1531.30 1136.80  583.3 1036.3 1280.2 1726.3  576.2
## 2  2102.80 2097.10 2137.60 1831.20 1607.9 1881.6 1849.9 2350.8 2108.0
## 3  1451.10 1957.90      NA      NA     NA 2411.4 1920.1 2821.3     NA
## 4  1245.20 1162.20 1661.40 1072.90  902.8 1024.5 1008.2 1188.1 1016.3
## 5  2001.85 1992.00 3282.65 1866.30 2386.1 2639.2 1727.2 2524.0 1448.2
## 6  1554.65 1487.15 1487.15 1399.15 1461.9 2005.6 1309.0 1736.8 1390.2
## 7   929.30  963.90 1084.00  751.10  694.9  807.1  879.5 1037.3  292.9
## 8  2034.30 2106.60 2764.80 1536.30 1605.8 2255.6 1719.2 2484.3 1299.2
## 9  1245.20 1162.20 1661.40 1072.90 2867.4 1486.4 1434.1 2204.7 1227.0
## 10 1744.60 1141.70 1457.50 1718.40 1369.1 2422.4 1885.5 1658.7 1039.6
##     a2014                    geom
## 1   845.9      POINT (-71.1 18.2)
## 2  1505.6 POINT (-69.63333 18.75)
## 3  1975.6    POINT (-69.9 19.633)
## 4   764.1      POINT (-70.7 18.9)
## 5  1928.7    POINT (-70.3 19.617)
## 6   908.9    POINT (-71.7 18.717)
## 7   502.0  POINT (-71.633 18.483)
## 8  1741.5    POINT (-70.55 19.75)
## 9  1812.5  POINT (-70.533 19.217)
## 10  909.4  POINT (-69.667 18.433)
\end{verbatim}

\begin{Shaded}
\begin{Highlighting}[]
\KeywordTok{st_layers}\NormalTok{(rutadiv)}
\end{Highlighting}
\end{Shaded}

\begin{verbatim}
## Driver: GPKG 
## Available layers:
##      layer_name geometry_type features fields
## 1 PROVCenso2010       Polygon       32      4
## 2  MUNCenso2010       Polygon      155      5
## 3  REGCenso2010       Polygon       10      2
\end{verbatim}

\begin{Shaded}
\begin{Highlighting}[]
\NormalTok{prov <-}\StringTok{ }\KeywordTok{st_read}\NormalTok{(rutadiv, }\DataTypeTok{layer =} \StringTok{'PROVCenso2010'}\NormalTok{)}
\end{Highlighting}
\end{Shaded}

\begin{verbatim}
## Reading layer `PROVCenso2010' from data source `/home/magda/unidad-0-asignacion-99-mi-proyecto-magdalenaviloriac/material-de-apoyo-master/data/divisionRD.gpkg' using driver `GPKG'
## Simple feature collection with 32 features and 4 fields
## geometry type:  MULTIPOLYGON
## dimension:      XY
## bbox:           xmin: 182215.8 ymin: 1933532 xmax: 571365.3 ymax: 2205216
## epsg (SRID):    32619
## proj4string:    +proj=utm +zone=19 +datum=WGS84 +units=m +no_defs
\end{verbatim}

\begin{Shaded}
\begin{Highlighting}[]
\NormalTok{prov}
\end{Highlighting}
\end{Shaded}

\begin{verbatim}
## Simple feature collection with 32 features and 4 fields
## geometry type:  MULTIPOLYGON
## dimension:      XY
## bbox:           xmin: 182215.8 ymin: 1933532 xmax: 571365.3 ymax: 2205216
## epsg (SRID):    32619
## proj4string:    +proj=utm +zone=19 +datum=WGS84 +units=m +no_defs
## First 10 features:
##    PROV REG         TOPONIMIA ENLACE                           geom
## 1    01  10 DISTRITO NACIONAL   1001 MULTIPOLYGON (((406845.9 20...
## 2    02  05              AZUA   0502 MULTIPOLYGON (((322129.5 20...
## 3    03  06           BAORUCO   0603 MULTIPOLYGON (((271940 2060...
## 4    04  06          BARAHONA   0604 MULTIPOLYGON (((291856.5 20...
## 5    05  04           DAJABÓN   0405 MULTIPOLYGON (((245433.3 21...
## 6    06  03            DUARTE   0306 MULTIPOLYGON (((374434.8 21...
## 7    07  07        ELÍAS PIÑA   0707 MULTIPOLYGON (((235630.8 21...
## 8    08  08          EL SEIBO   0808 MULTIPOLYGON (((523436.4 20...
## 9    09  01         ESPAILLAT   0109 MULTIPOLYGON (((385993.5 21...
## 10   10  06     INDEPENDENCIA   0610 MULTIPOLYGON (((205698.2 20...
\end{verbatim}

Exploremos el CRS del objeto \texttt{obs}.

\begin{Shaded}
\begin{Highlighting}[]
\KeywordTok{st_crs}\NormalTok{(pre)}
\end{Highlighting}
\end{Shaded}

\begin{verbatim}
## Coordinate Reference System:
##   EPSG: 4326 
##   proj4string: "+proj=longlat +datum=WGS84 +no_defs"
\end{verbatim}

Transformemos a 32619:

\begin{Shaded}
\begin{Highlighting}[]
\NormalTok{crsdestino <-}\StringTok{ }\DecValTok{32619}
\NormalTok{preutm <-}\StringTok{ }\NormalTok{pre }\OperatorTok\StringTok{ }\KeywordTok{st_transform}\NormalTok{(}\DataTypeTok{crs =}\NormalTok{ crsdestino)}
\NormalTok{preutm}
\end{Highlighting}
\end{Shaded}

\begin{verbatim}
## Simple feature collection with 25 features and 37 fields
## geometry type:  POINT
## dimension:      XY
## bbox:           xmin: 215264.1 ymin: 1999092 xmax: 566794.7 ymax: 2197035
## epsg (SRID):    32619
## proj4string:    +proj=utm +zone=19 +datum=WGS84 +units=m +no_defs
## First 10 features:
##            Estación  a1979  a1980  a1981  a1982  a1983  a1984  a1985
## 1          Barahona 1740.0 1053.6 1435.3  815.3 1183.0  584.1  997.8
## 2         Bayaguana 2794.3 1761.5 2412.4 1758.6 1857.1 1645.6 1928.3
## 3           Cabrera 2035.0 1276.8     NA 2136.9 1703.8 1888.7 1557.1
## 4         Constanza 1652.1 1166.9 1343.3  921.2  828.4     NA  892.8
## 5  Gaspar Hernández     NA 1443.8 2174.9 1844.1 1688.8 2208.8 1895.5
## 6       Hondo Valle 1823.6 1778.2 2203.7 1709.9 1841.3 1796.6 1309.5
## 7            Jimaní 1060.7  639.1  960.2  507.5  610.7  641.5  689.6
## 8          La Unión 1781.5 1630.6 2304.4 1413.1 1288.4 1499.4 1157.1
## 9           La Vega 1833.5 1304.3 1993.7 1483.2 1353.9 1550.1 1084.9
## 10     Las Américas 1958.4  958.7 1513.4  787.4  975.5  954.9 1398.2
##     a1986  a1987  a1988  a1989   a1990  a1991  a1992  a1993  a1994   a1995
## 1  1080.0 1423.9  704.7 1011.6 1075.20  983.1 1112.5  968.5 1622.4  956.00
## 2  2182.2 2273.5 1813.2 1730.6 1823.40 1850.3 1765.7 1606.2 1892.8 1360.10
## 3  1597.0 2059.7     NA 1176.9 1183.40  957.6     NA     NA     NA      NA
## 4   715.8  786.9  837.7  671.5  875.35     NA  858.6  858.6  900.7  839.40
## 5  2874.7 2360.8 1426.3 1214.2 1530.70     NA 1257.5 1345.3 1824.9 1665.45
## 6  1589.7 1778.8 1766.5 1722.8 1596.10 1088.4 1731.0 1887.0 1772.0 1288.30
## 7   802.4  648.9  521.0  680.7  880.00  311.6  809.2  472.9  840.2  909.00
## 8  1313.1 1786.5 1888.8 1222.8 1808.00 1250.4 1555.2 1484.8 1035.9  877.70
## 9  1767.1 1663.2 1934.9 1192.4 1664.40 1146.4 1565.6 1855.4 1455.7 1175.40
## 10 1419.0 1866.4 1620.5 1151.7      NA  997.0     NA     NA     NA 1017.50
##      a1996   a1997  a1998  a1999  a2000  a2001  a2002   a2003  a2004
## 1   965.65  662.60  684.6  662.7  600.0  600.0  997.6  942.60  972.6
## 2  1867.70 1618.60 2156.6 1712.5 1868.5 1796.1 1658.0 2117.30 1554.2
## 3       NA      NA     NA     NA 1538.6 1852.9  946.9 1810.95 2053.3
## 4  1167.30  525.10 1492.7 1077.8  951.3  787.1  959.2 1084.10  985.9
## 5  2656.80  984.80 2147.9 1791.9 1716.9 2178.8 1093.4 2058.50 1906.8
## 6  1447.90  912.65 1813.9 1762.2 2285.9 1604.3 1477.4 1628.10 1617.7
## 7   816.20  358.20  824.1 1037.0  833.9  488.4  510.1  656.70  866.9
## 8  1980.50  554.20 1744.1 1314.3 1148.5 1360.5  972.1 1802.00 2550.1
## 9  1772.50 1018.80 1549.6 1817.9 1368.6 1522.0 1200.7 2290.60 1825.7
## 10 1019.60  651.20 1218.6 1125.9  809.7  747.6  933.4 1083.60 1338.9
##      a2005   a2006   a2007   a2008  a2009  a2010  a2011  a2012  a2013
## 1  1274.60 1118.40 1531.30 1136.80  583.3 1036.3 1280.2 1726.3  576.2
## 2  2102.80 2097.10 2137.60 1831.20 1607.9 1881.6 1849.9 2350.8 2108.0
## 3  1451.10 1957.90      NA      NA     NA 2411.4 1920.1 2821.3     NA
## 4  1245.20 1162.20 1661.40 1072.90  902.8 1024.5 1008.2 1188.1 1016.3
## 5  2001.85 1992.00 3282.65 1866.30 2386.1 2639.2 1727.2 2524.0 1448.2
## 6  1554.65 1487.15 1487.15 1399.15 1461.9 2005.6 1309.0 1736.8 1390.2
## 7   929.30  963.90 1084.00  751.10  694.9  807.1  879.5 1037.3  292.9
## 8  2034.30 2106.60 2764.80 1536.30 1605.8 2255.6 1719.2 2484.3 1299.2
## 9  1245.20 1162.20 1661.40 1072.90 2867.4 1486.4 1434.1 2204.7 1227.0
## 10 1744.60 1141.70 1457.50 1718.40 1369.1 2422.4 1885.5 1658.7 1039.6
##     a2014                     geom
## 1   845.9 POINT (277900.2 2013585)
## 2  1505.6 POINT (433242.1 2073284)
## 3  1975.6   POINT (405636 2171119)
## 4   764.1 POINT (320947.7 2090623)
## 5  1928.7 POINT (363678.2 2169619)
## 6   908.9 POINT (215264.1 2071669)
## 7   502.0 POINT (221953.7 2045651)
## 8  1741.5 POINT (337592.1 2184559)
## 9  1812.5 POINT (338847.1 2125548)
## 10  909.4 POINT (429562.7 2038222)
\end{verbatim}

\subsubsection{EDA básico}\label{eda-buxe1sico}

Obtengamos los estadísticos básicos para el año 1998:

\begin{Shaded}
\begin{Highlighting}[]
\KeywordTok{nrow}\NormalTok{(preutm)}
\end{Highlighting}
\end{Shaded}

\begin{verbatim}
## [1] 25
\end{verbatim}

\begin{Shaded}
\begin{Highlighting}[]
\KeywordTok{summary}\NormalTok{(preutm}\OperatorTok{$}\NormalTok{a1998)}
\end{Highlighting}
\end{Shaded}

\begin{verbatim}
##    Min. 1st Qu.  Median    Mean 3rd Qu.    Max.    NA's 
##   684.6  1151.7  1580.5  1644.0  1987.7  3011.3       2
\end{verbatim}

\begin{Shaded}
\begin{Highlighting}[]
\KeywordTok{hist}\NormalTok{(preutm}\OperatorTok{$}\NormalTok{a1998)}
\end{Highlighting}
\end{Shaded}

\includegraphics[width=600px]{proyecto_files/figure-latex/esda-1998-1}

\begin{Shaded}
\begin{Highlighting}[]
\KeywordTok{hist}\NormalTok{(}\KeywordTok{log}\NormalTok{(preutm}\OperatorTok{$}\NormalTok{a1998))}
\end{Highlighting}
\end{Shaded}

\includegraphics[width=600px]{proyecto_files/figure-latex/esda-1998-2}

\begin{Shaded}
\begin{Highlighting}[]
\KeywordTok{shapiro.test}\NormalTok{(preutm}\OperatorTok{$}\NormalTok{a1998)}
\end{Highlighting}
\end{Shaded}

\begin{verbatim}
## 
##  Shapiro-Wilk normality test
## 
## data:  preutm$a1998
## W = 0.94806, p-value = 0.2666
\end{verbatim}

\begin{Shaded}
\begin{Highlighting}[]
\KeywordTok{shapiro.test}\NormalTok{(}\KeywordTok{log}\NormalTok{(pre}\OperatorTok{$}\NormalTok{a1998))}
\end{Highlighting}
\end{Shaded}

\begin{verbatim}
## 
##  Shapiro-Wilk normality test
## 
## data:  log(pre$a1998)
## W = 0.97788, p-value = 0.8671
\end{verbatim}

Como vemos, los datos siguen distribución normal, pero hay algo de sesgo
hacia la derecha en la distribución. Igualmente, de los 25 observatorios
que hay en todo el país, para 1998 en al menos 2 hay datos perdidos
(\texttt{NA}). Eliminemos dichos observatorios, generemos un objeto que
incluya solamente a 1998 y que contenga igualmente una columna con datos
transformados:

\begin{Shaded}
\begin{Highlighting}[]
\NormalTok{pre1998 <-}\StringTok{ }\KeywordTok{na.omit}\NormalTok{(preutm[,}\KeywordTok{c}\NormalTok{(}\StringTok{'Estación', '}\NormalTok{a1998}\StringTok{')])}
\StringTok{pre1998$a1998log <- log(pre1998$a1998)}
\StringTok{pre1998}
\end{Highlighting}
\end{Shaded}

\begin{verbatim}
## Simple feature collection with 23 features and 3 fields
## geometry type:  POINT
## dimension:      XY
## bbox:           xmin: 215264.1 ymin: 1999092 xmax: 566794.7 ymax: 2197035
## epsg (SRID):    32619
## proj4string:    +proj=utm +zone=19 +datum=WGS84 +units=m +no_defs
## First 10 features:
##            Estación  a1998                     geom a1998log
## 1          Barahona  684.6 POINT (277900.2 2013585) 6.528835
## 2         Bayaguana 2156.6 POINT (433242.1 2073284) 7.676288
## 4         Constanza 1492.7 POINT (320947.7 2090623) 7.308342
## 5  Gaspar Hernández 2147.9 POINT (363678.2 2169619) 7.672246
## 6       Hondo Valle 1813.9 POINT (215264.1 2071669) 7.503235
## 7            Jimaní  824.1 POINT (221953.7 2045651) 6.714292
## 8          La Unión 1744.1 POINT (337592.1 2184559) 7.463994
## 9           La Vega 1549.6 POINT (338847.1 2125548) 7.345752
## 10     Las Américas 1218.6 POINT (429562.7 2038222) 7.105458
## 11             Moca 1036.4 POINT (342475.8 2143891) 6.943508
\end{verbatim}

Representemos los observatorios, estilizando por tono según la
precipitación del año 1998:

\begin{Shaded}
\begin{Highlighting}[]
\KeywordTok{library}\NormalTok{(ggplot2)}
\KeywordTok{ggplot}\NormalTok{() }\OperatorTok{+}
\StringTok{  }\KeywordTok{geom_sf}\NormalTok{(}\DataTypeTok{data =}\NormalTok{ prov, }\DataTypeTok{fill =} \StringTok{'white'}\NormalTok{) }\OperatorTok{+}
\StringTok{  }\KeywordTok{geom_sf}\NormalTok{(}\DataTypeTok{data =}\NormalTok{ pre1998, }\KeywordTok{aes}\NormalTok{(}\DataTypeTok{col =}\NormalTok{ a1998log), }\DataTypeTok{size =} \DecValTok{6}\NormalTok{) }\OperatorTok{+}
\StringTok{  }\KeywordTok{scale_colour_gradient}\NormalTok{(}\DataTypeTok{low=}\StringTok{"#deebf7"}\NormalTok{, }\DataTypeTok{high=}\StringTok{"#3182bd"}\NormalTok{) }\OperatorTok{+}
\StringTok{  }\KeywordTok{geom_sf_text}\NormalTok{(}\DataTypeTok{data =}\NormalTok{ prov, }\KeywordTok{aes}\NormalTok{(}\DataTypeTok{label=}\NormalTok{TOPONIMIA), }\DataTypeTok{check_overlap =}\NormalTok{ T, }\DataTypeTok{size =} \DecValTok{2}\NormalTok{) }\OperatorTok{+}
\StringTok{  }\KeywordTok{geom_sf_text}\NormalTok{(}\DataTypeTok{data =}\NormalTok{ pre1998, }\KeywordTok{aes}\NormalTok{(}\DataTypeTok{label=}\NormalTok{Estación), }\DataTypeTok{check_overlap =}\NormalTok{ T, }\DataTypeTok{size =} \FloatTok{1.5}\NormalTok{) }\OperatorTok{+}
\StringTok{  }\KeywordTok{theme_bw}\NormalTok{()}
\end{Highlighting}
\end{Shaded}

\includegraphics{proyecto_files/figure-latex/mapa-pre-1998-1.pdf} \#\#\#
Variograma muestral

Generemos el variograma muestral para el logaritmo de la precipitación.
Para ello empleamos la función \texttt{variogram}.

\begin{Shaded}
\begin{Highlighting}[]
\NormalTok{v98 <-}\StringTok{ }\KeywordTok{variogram}\NormalTok{(a1998log}\OperatorTok{~}\DecValTok{1}\NormalTok{, pre1998)}
\NormalTok{v98}
\end{Highlighting}
\end{Shaded}

\begin{verbatim}
##    np       dist      gamma dir.hor dir.ver   id
## 1   1   8896.559 0.08905032       0       0 var1
## 2   6  22368.506 0.13141962       0       0 var1
## 3  10  32110.478 0.08693293       0       0 var1
## 4   6  40706.420 0.03361319       0       0 var1
## 5   6  50780.415 0.12041650       0       0 var1
## 6  12  58446.995 0.10567299       0       0 var1
## 7   8  67239.009 0.06443833       0       0 var1
## 8   7  77115.401 0.11793609       0       0 var1
## 9  17  85657.115 0.15764350       0       0 var1
## 10 12  93304.363 0.11849679       0       0 var1
## 11  7 103800.452 0.02103140       0       0 var1
## 12 19 112257.676 0.11604918       0       0 var1
## 13 16 120305.537 0.21044124       0       0 var1
## 14 12 128383.382 0.15380975       0       0 var1
\end{verbatim}

\begin{Shaded}
\begin{Highlighting}[]
\KeywordTok{plot}\NormalTok{(v98, }\DataTypeTok{plot.numbers =}\NormalTok{ T)}
\end{Highlighting}
\end{Shaded}

\includegraphics[width=800px]{proyecto_files/figure-latex/vgm-pre1979-1}

Nótese la fórmula \texttt{a1998log\textasciitilde{}1}, la cual indica
que la precipitación de 1998 es la variable sobre la cual se generará el
variograma contra un modelo de media, que en este caso es simplemente un
intercepto (media desconocida y constante). Típicamente, este variograma
servirá para realizar un kriging ordinario.

La función \texttt{variogram} fija una distancia máxima de búsqueda
(\texttt{cutoff}), que equivale a un tercio de la diagonal del recuadro
delimitador (\emph{bounding box}), y fija intervalos de anchura
constante (\texttt{width}, que es la distancia de los intervalos
\emph{hi}, referida anteriormente) equivalentes a \texttt{cutoff/15}.
Dichos parámetros, \texttt{cutoff} y \texttt{width} pueden modificarse
por argumentos dentro de la función \texttt{variogram}.

\subsubsection{Variograma modelo}\label{variograma-modelo}

A partir del variograma muestral, generamos un variograma modelo que
será el que utlizará la función \texttt{krige} para realizar la
interpolación. Probamos varias opciones en función de lo visto en el
variograma muestral.

\begin{Shaded}
\begin{Highlighting}[]
\NormalTok{v98_m <-}\StringTok{ }\KeywordTok{fit.variogram}\NormalTok{(v98, }\KeywordTok{vgm}\NormalTok{(}\DataTypeTok{model =} \StringTok{"Sph"}\NormalTok{, }\DataTypeTok{range =} \DecValTok{50000}\NormalTok{))}
\NormalTok{v98_m}
\end{Highlighting}
\end{Shaded}

\begin{verbatim}
##   model     psill    range
## 1   Sph 0.1078617 13982.71
\end{verbatim}

\begin{Shaded}
\begin{Highlighting}[]
\KeywordTok{plot}\NormalTok{(v98, v98_m, }\DataTypeTok{plot.numbers =}\NormalTok{ T)}
\end{Highlighting}
\end{Shaded}

\includegraphics[width=800px]{proyecto_files/figure-latex/vgm-pre1998-ajus-exp-1}

\begin{Shaded}
\begin{Highlighting}[]
\NormalTok{v98_m2 <-}\StringTok{ }\KeywordTok{fit.variogram}\NormalTok{(v98, }\KeywordTok{vgm}\NormalTok{(}\DataTypeTok{model =} \StringTok{"Exp"}\NormalTok{, }\DataTypeTok{range =} \DecValTok{50000}\NormalTok{))}
\NormalTok{v98_m2}
\end{Highlighting}
\end{Shaded}

\begin{verbatim}
##   model     psill    range
## 1   Exp 0.1073634 4605.641
\end{verbatim}

\begin{Shaded}
\begin{Highlighting}[]
\KeywordTok{plot}\NormalTok{(v98, v98_m2, }\DataTypeTok{plot.numbers =}\NormalTok{ T)}
\end{Highlighting}
\end{Shaded}

\includegraphics[width=800px]{proyecto_files/figure-latex/vgm-pre1998-ajus-exp-2}

\begin{Shaded}
\begin{Highlighting}[]
\NormalTok{v98_m3 <-}\StringTok{ }\KeywordTok{fit.variogram}\NormalTok{(v98, }\KeywordTok{vgm}\NormalTok{(}\DataTypeTok{model =} \StringTok{"Gau"}\NormalTok{, }\DataTypeTok{range =} \DecValTok{50000}\NormalTok{))}
\end{Highlighting}
\end{Shaded}

\begin{verbatim}
## Warning in fit.variogram(v98, vgm(model = "Gau", range = 50000)): No
## convergence after 200 iterations: try different initial values?
\end{verbatim}

\begin{Shaded}
\begin{Highlighting}[]
\NormalTok{v98_m3}
\end{Highlighting}
\end{Shaded}

\begin{verbatim}
##   model     psill   range
## 1   Gau 0.1078316 5250.27
\end{verbatim}

\begin{Shaded}
\begin{Highlighting}[]
\KeywordTok{plot}\NormalTok{(v98, v98_m3, }\DataTypeTok{plot.numbers =}\NormalTok{ T)}
\end{Highlighting}
\end{Shaded}

\includegraphics[width=800px]{proyecto_files/figure-latex/vgm-pre1998-ajus-exp-3}

\begin{Shaded}
\begin{Highlighting}[]
\KeywordTok{attr}\NormalTok{(v98_m, }\StringTok{'SSErr'}\NormalTok{)}
\end{Highlighting}
\end{Shaded}

\begin{verbatim}
## [1] 5.877132e-11
\end{verbatim}

\begin{Shaded}
\begin{Highlighting}[]
\KeywordTok{attr}\NormalTok{(v98_m2, }\StringTok{'SSErr'}\NormalTok{) }\CommentTok{#Elegimos este}
\end{Highlighting}
\end{Shaded}

\begin{verbatim}
## [1] 5.93196e-11
\end{verbatim}

\begin{Shaded}
\begin{Highlighting}[]
\KeywordTok{attr}\NormalTok{(v98_m3, }\StringTok{'SSErr'}\NormalTok{)}
\end{Highlighting}
\end{Shaded}

\begin{verbatim}
## [1] 6.080129e-11
\end{verbatim}

\subsubsection{Interpolación por kriging
ordinario}\label{interpolaciuxf3n-por-kriging-ordinario}

Antes de realizar la interpolación, necesitamos una cuadrícula que
``llenaremos'' con las predicciones. Creemos una cuadrícula para RD, en
este caso, de baja resolución, 10x10km:

\begin{Shaded}
\begin{Highlighting}[]
\KeywordTok{library}\NormalTok{(stars)}
\end{Highlighting}
\end{Shaded}

\begin{verbatim}
## Loading required package: abind
\end{verbatim}

\begin{Shaded}
\begin{Highlighting}[]
\NormalTok{grd <-}\StringTok{ }\KeywordTok{st_bbox}\NormalTok{(prov) }\OperatorTok
\StringTok{  }\KeywordTok{st_as_stars}\NormalTok{(}\DataTypeTok{dx =} \DecValTok{10000}\NormalTok{) }\OperatorTok\StringTok{ }\CommentTok{#10000 metros=10km de resolución espacial}
\StringTok{  }\KeywordTok{st_set_crs}\NormalTok{(crsdestino) }\OperatorTok
\StringTok{  }\KeywordTok{st_crop}\NormalTok{(prov)}
\NormalTok{grd}
\end{Highlighting}
\end{Shaded}

\begin{verbatim}
## stars object with 2 dimensions and 1 attribute
## attribute(s):
##     values    
##  Min.   :0    
##  1st Qu.:0    
##  Median :0    
##  Mean   :0    
##  3rd Qu.:0    
##  Max.   :0    
##  NA's   :605  
## dimension(s):
##   from to  offset  delta                       refsys point values    
## x    1 39  182216  10000 +proj=utm +zone=19 +datum...    NA   NULL [x]
## y    1 28 2205216 -10000 +proj=utm +zone=19 +datum...    NA   NULL [y]
\end{verbatim}

\begin{Shaded}
\begin{Highlighting}[]
\KeywordTok{plot}\NormalTok{(grd)}
\end{Highlighting}
\end{Shaded}

\includegraphics[width=800px]{proyecto_files/figure-latex/grd-1}

Sobre ella, ejecutamos la interpolación por kriging ordinario. La
función \texttt{krige} asume que se trata de kriging ordinario, dado que
no se especifica un valor para el argumento \texttt{beta}, o media.

\begin{Shaded}
\begin{Highlighting}[]
\NormalTok{k <-}\StringTok{ }\KeywordTok{krige}\NormalTok{(}\DataTypeTok{formula =}\NormalTok{ a1998log}\OperatorTok{~}\DecValTok{1}\NormalTok{, }\DataTypeTok{locations =}\NormalTok{ pre1998, }\DataTypeTok{newdata =}\NormalTok{ grd, }\DataTypeTok{model =}\NormalTok{ v98_m2)}
\end{Highlighting}
\end{Shaded}

\begin{verbatim}
## [using ordinary kriging]
\end{verbatim}

\begin{Shaded}
\begin{Highlighting}[]
\NormalTok{k}
\end{Highlighting}
\end{Shaded}

\begin{verbatim}
## stars object with 2 dimensions and 2 attributes
## attribute(s):
##    var1.pred       var1.var      
##  Min.   :6.951   Min.   :0.0542  
##  1st Qu.:7.330   1st Qu.:0.1115  
##  Median :7.331   Median :0.1120  
##  Mean   :7.332   Mean   :0.1103  
##  3rd Qu.:7.333   3rd Qu.:0.1121  
##  Max.   :7.641   Max.   :0.1121  
##  NA's   :605     NA's   :605     
## dimension(s):
##   from to  offset  delta                       refsys point values    
## x    1 39  182216  10000 +proj=utm +zone=19 +datum...    NA   NULL [x]
## y    1 28 2205216 -10000 +proj=utm +zone=19 +datum...    NA   NULL [y]
\end{verbatim}

El objeto \texttt{k} es un ráster \texttt{stars} con dos variables,
\texttt{var1.pred} y \texttt{var1.var}, que son, respectivamente, la
predicción y la varianza de la predicción. La función \texttt{plot}
contiene un método para imprimir el objeto \texttt{k}.

\begin{Shaded}
\begin{Highlighting}[]
\KeywordTok{plot}\NormalTok{(k)}
\end{Highlighting}
\end{Shaded}

\includegraphics[width=800px]{proyecto_files/figure-latex/krige-plot-raw-1}

Utilicemos \texttt{ggplot} para representar el objeto \texttt{stars}.

\begin{Shaded}
\begin{Highlighting}[]
\KeywordTok{ggplot}\NormalTok{() }\OperatorTok{+}
\StringTok{  }\KeywordTok{geom_stars}\NormalTok{(}\DataTypeTok{data =}\NormalTok{ k, }\KeywordTok{aes}\NormalTok{(}\DataTypeTok{fill =}\NormalTok{ var1.pred, }\DataTypeTok{x =}\NormalTok{ x, }\DataTypeTok{y =}\NormalTok{ y)) }\OperatorTok{+}\StringTok{ }
\StringTok{  }\KeywordTok{scale_fill_gradient}\NormalTok{(}\DataTypeTok{low=}\StringTok{"#deebf7"}\NormalTok{, }\DataTypeTok{high=}\StringTok{"#3182bd"}\NormalTok{) }\OperatorTok{+}
\StringTok{  }\KeywordTok{geom_sf}\NormalTok{(}\DataTypeTok{data =} \KeywordTok{st_cast}\NormalTok{(prov, }\StringTok{"MULTILINESTRING"}\NormalTok{)) }\OperatorTok{+}
\StringTok{  }\KeywordTok{geom_sf}\NormalTok{(}\DataTypeTok{data =}\NormalTok{ pre1998) }\OperatorTok{+}
\StringTok{  }\KeywordTok{geom_sf_text}\NormalTok{(}\DataTypeTok{data =}\NormalTok{ prov, }\KeywordTok{aes}\NormalTok{(}\DataTypeTok{label=}\NormalTok{TOPONIMIA), }\DataTypeTok{check_overlap =}\NormalTok{ T, }\DataTypeTok{size =} \DecValTok{2}\NormalTok{) }\OperatorTok{+}
\StringTok{  }\KeywordTok{theme_bw}\NormalTok{()}
\end{Highlighting}
\end{Shaded}

\includegraphics{proyecto_files/figure-latex/krige-log-1.pdf} Nótese en
la leyenda que el objeto \texttt{k}, variable \texttt{var1.pred}
contiene las predicciones del logaritmo de la precipitación para la
cuadrícula de 10x10km (de ahí que el rango de la leyenda sea
\texttt{6.8-8.0}). Si calculamos \emph{e6.8} obtendremos el valor de
precipitación del límite inferior, y si calculamos \emph{e8} obtendremos
el límite superior.

Si queremos representar los valores de precipitación, debemos realizar
la operación inversa, que sería elevar al \texttt{e} el valor predicho
en \texttt{k}, lo cual se realiza mediante la función \texttt{exp()}.

\begin{Shaded}
\begin{Highlighting}[]
\KeywordTok{ggplot}\NormalTok{() }\OperatorTok{+}
\StringTok{  }\KeywordTok{geom_stars}\NormalTok{(}\DataTypeTok{data =} \KeywordTok{exp}\NormalTok{(k), }\KeywordTok{aes}\NormalTok{(}\DataTypeTok{fill =}\NormalTok{ var1.pred, }\DataTypeTok{x =}\NormalTok{ x, }\DataTypeTok{y =}\NormalTok{ y)) }\OperatorTok{+}\StringTok{ }
\StringTok{  }\KeywordTok{scale_fill_gradient}\NormalTok{(}\DataTypeTok{low=}\StringTok{"#deebf7"}\NormalTok{, }\DataTypeTok{high=}\StringTok{"#3182bd"}\NormalTok{, }\DataTypeTok{trans =} \StringTok{'log10'}\NormalTok{) }\OperatorTok{+}
\StringTok{  }\KeywordTok{geom_sf}\NormalTok{(}\DataTypeTok{data =} \KeywordTok{st_cast}\NormalTok{(prov, }\StringTok{"MULTILINESTRING"}\NormalTok{)) }\OperatorTok{+}
\StringTok{  }\KeywordTok{geom_sf}\NormalTok{(}\DataTypeTok{data =}\NormalTok{ pre1998) }\OperatorTok{+}
\StringTok{  }\KeywordTok{geom_sf_text}\NormalTok{(}\DataTypeTok{data =}\NormalTok{ prov, }\KeywordTok{aes}\NormalTok{(}\DataTypeTok{label=}\NormalTok{TOPONIMIA), }\DataTypeTok{check_overlap =}\NormalTok{ T, }\DataTypeTok{size =} \DecValTok{2}\NormalTok{) }\OperatorTok{+}
\StringTok{  }\KeywordTok{theme_bw}\NormalTok{()}
\end{Highlighting}
\end{Shaded}

\includegraphics{proyecto_files/figure-latex/krige-1.pdf} \#\# Estudio
de caso: temperatura de República Dominicana mediante kriging universal

Hasta este punto, logramos ejecutar un kriging ordinario para predecir
el valor de la precipitación de 1998 para todo el país a partir de 21
observatorios. Notemos que se trataba de un kriging ordinario, porque a
la función \texttt{krige} no le introducimos una media (argumento
\texttt{beta}), e igualmente porque con la función \texttt{variogram}
generamos un variograma contra un intercepto (fórmula
\texttt{a1998log\textasciitilde{}1}).

El kriging universal predice el valor de la variable de interés en
función del modelo espacial aportado por el variograma Y, al mismo
tiempo, considerando covariables mediante polinomios. En este ejemplo,
tomaremos la temperatura registrada en observatorios de ONAMET.

\subsubsection{Datos fuente}\label{datos-fuente-1}

\begin{Shaded}
\begin{Highlighting}[]
\NormalTok{rutatemp <-}\StringTok{ 'material-de-apoyo-master/data/onamet_temp_anual.gpkg'}
\KeywordTok{st_layers}\NormalTok{(rutatemp)}
\end{Highlighting}
\end{Shaded}

\begin{verbatim}
## Driver: GPKG 
## Available layers:
##          layer_name geometry_type features fields
## 1 onamet_temp_anual         Point       72     14
\end{verbatim}

\begin{Shaded}
\begin{Highlighting}[]
\NormalTok{temp <-}\StringTok{ }\KeywordTok{st_read}\NormalTok{(rutatemp)}
\end{Highlighting}
\end{Shaded}

\begin{verbatim}
## Reading layer `onamet_temp_anual' from data source `/home/magda/unidad-0-asignacion-99-mi-proyecto-magdalenaviloriac/material-de-apoyo-master/data/onamet_temp_anual.gpkg' using driver `GPKG'
## Simple feature collection with 72 features and 14 fields
## geometry type:  POINT
## dimension:      XY
## bbox:           xmin: 199028.4 ymin: 1967717 xmax: 566825.1 ymax: 2199684
## epsg (SRID):    32619
## proj4string:    +proj=utm +zone=19 +datum=WGS84 +units=m +no_defs
\end{verbatim}

\begin{Shaded}
\begin{Highlighting}[]
\NormalTok{temp}
\end{Highlighting}
\end{Shaded}

\begin{verbatim}
## Simple feature collection with 72 features and 14 fields
## geometry type:  POINT
## dimension:      XY
## bbox:           xmin: 199028.4 ymin: 1967717 xmax: 566825.1 ymax: 2199684
## epsg (SRID):    32619
## proj4string:    +proj=utm +zone=19 +datum=WGS84 +units=m +no_defs
## First 10 features:
##           nombre  ene  feb  mar  abr  may  jun  jul  ago  sep  oct  nov
## 1        HERRERA 24.6 24.4 24.8 25.8 26.4 27.1 27.4 27.4 27.1 26.9 26.1
## 2       LA UNION 23.3 23.3 23.8 24.7 25.7 27.0 27.0 27.1 26.9 26.3 25.0
## 3  ARROYO BARRIL 24.6 24.9 25.6 26.1 26.4 27.1 27.1 26.8 26.8 26.6 25.7
## 4           AZUA 25.2 25.5 26.2 26.8 27.3 27.7 28.4 28.5 28.1 27.4 26.6
## 5           BANI 25.9 26.0 26.7 27.4 27.6 27.9 28.6 28.5 28.1 27.6 27.0
## 6       BARAHONA 24.6 24.8 25.4 26.2 26.7 27.3 27.9 27.9 27.5 26.7 26.2
## 7      BAYAGUANA 24.6 24.9 25.6 26.3 27.0 27.6 27.7 27.6 27.5 27.2 26.3
## 8          BONAO 23.4 23.8 24.6 25.4 25.9 26.8 26.9 27.0 26.8 26.4 25.2
## 9        CABRERA 24.7 24.8 25.2 25.6 26.0 26.6 26.8 26.8 26.7 26.5 25.7
## 10       CEVICOS 23.0 23.4 24.3 25.2 26.0 26.6 26.6 26.6 26.6 26.1 24.7
##     dic tanual                     geom
## 1  25.1   26.1 POINT (397885.6 2042020)
## 2  23.7   25.3 POINT (337547.6 2184493)
## 3  25.0   26.1 POINT (452651.8 2124797)
## 4  25.5   26.9 POINT (316911.7 2040776)
## 5  25.9   27.3 POINT (359011.1 2020133)
## 6  25.2   26.4 POINT (277856.2 2013510)
## 7  25.0   26.4 POINT (433198.1 2073212)
## 8  23.8   25.5 POINT (352534.2 2093960)
## 9  24.9   25.9 POINT (405589.5 2171085)
## 10 23.3   25.2 POINT (398204.6 2101037)
\end{verbatim}

Exploremos el CRS del objeto \texttt{obs}.

\begin{Shaded}
\begin{Highlighting}[]
\KeywordTok{st_crs}\NormalTok{(temp)}
\end{Highlighting}
\end{Shaded}

\begin{verbatim}
## Coordinate Reference System:
##   EPSG: 32619 
##   proj4string: "+proj=utm +zone=19 +datum=WGS84 +units=m +no_defs"
\end{verbatim}

Dado que es EPSG:32619 no necesitamos realizar transformación alguna.

\subsubsection{EDA básico}\label{eda-buxe1sico-1}

Obtengamos los estadísticos básicos del objeto \texttt{temp} y de su
variable \texttt{tanual}:

\begin{Shaded}
\begin{Highlighting}[]
\KeywordTok{nrow}\NormalTok{(temp)}
\end{Highlighting}
\end{Shaded}

\begin{verbatim}
## [1] 72
\end{verbatim}

\begin{Shaded}
\begin{Highlighting}[]
\KeywordTok{summary}\NormalTok{(temp}\OperatorTok{$}\NormalTok{tanual)}
\end{Highlighting}
\end{Shaded}

\begin{verbatim}
##    Min. 1st Qu.  Median    Mean 3rd Qu.    Max. 
##   18.20   25.20   25.90   25.50   26.43   28.40
\end{verbatim}

\begin{Shaded}
\begin{Highlighting}[]
\KeywordTok{hist}\NormalTok{(temp}\OperatorTok{$}\NormalTok{tanual)}
\end{Highlighting}
\end{Shaded}

\includegraphics[width=600px]{proyecto_files/figure-latex/esda-temp-1}

\begin{Shaded}
\begin{Highlighting}[]
\KeywordTok{qqnorm}\NormalTok{(temp}\OperatorTok{$}\NormalTok{tanual)}
\end{Highlighting}
\end{Shaded}

\includegraphics[width=600px]{proyecto_files/figure-latex/esda-temp-2}

\begin{Shaded}
\begin{Highlighting}[]
\KeywordTok{hist}\NormalTok{(}\KeywordTok{log}\NormalTok{(temp}\OperatorTok{$}\NormalTok{tanual))}
\end{Highlighting}
\end{Shaded}

\includegraphics[width=600px]{proyecto_files/figure-latex/esda-temp-3}

\begin{Shaded}
\begin{Highlighting}[]
\KeywordTok{qqnorm}\NormalTok{(}\KeywordTok{log}\NormalTok{(temp}\OperatorTok{$}\NormalTok{tanual))}
\end{Highlighting}
\end{Shaded}

\includegraphics[width=600px]{proyecto_files/figure-latex/esda-temp-4}

\begin{Shaded}
\begin{Highlighting}[]
\KeywordTok{shapiro.test}\NormalTok{(temp}\OperatorTok{$}\NormalTok{tanual)}
\end{Highlighting}
\end{Shaded}

\begin{verbatim}
## 
##  Shapiro-Wilk normality test
## 
## data:  temp$tanual
## W = 0.81613, p-value = 4.783e-08
\end{verbatim}

\begin{Shaded}
\begin{Highlighting}[]
\KeywordTok{shapiro.test}\NormalTok{(}\KeywordTok{log}\NormalTok{(temp}\OperatorTok{$}\NormalTok{tanual))}
\end{Highlighting}
\end{Shaded}

\begin{verbatim}
## 
##  Shapiro-Wilk normality test
## 
## data:  log(temp$tanual)
## W = 0.77059, p-value = 3e-09
\end{verbatim}

Dado que en este caso existe una fuerte desviación de una distribución
normal, debemos tenerlo en cuenta al modelizar la temperatura respecto
de la elevación. Al menos los residuos deberían tener distribución
normal. Exploraremos el modelo oportunamente. Visualicemos los datos en
un mapa

\begin{Shaded}
\begin{Highlighting}[]
\KeywordTok{library}\NormalTok{(RColorBrewer)}
\KeywordTok{ggplot}\NormalTok{() }\OperatorTok{+}
\StringTok{  }\KeywordTok{geom_sf}\NormalTok{(}\DataTypeTok{data =}\NormalTok{ prov, }\DataTypeTok{fill =} \StringTok{'white'}\NormalTok{) }\OperatorTok{+}
\StringTok{  }\KeywordTok{geom_sf}\NormalTok{(}\DataTypeTok{data =}\NormalTok{ temp, }\KeywordTok{aes}\NormalTok{(}\DataTypeTok{col =}\NormalTok{ tanual), }\DataTypeTok{size =} \DecValTok{6}\NormalTok{) }\OperatorTok{+}\StringTok{ }
\StringTok{  }\KeywordTok{scale_colour_gradientn}\NormalTok{(}\DataTypeTok{colours =} \KeywordTok{rev}\NormalTok{(}\KeywordTok{brewer.pal}\NormalTok{(}\DecValTok{9}\NormalTok{, }\DataTypeTok{name =} \StringTok{'RdBu'}\NormalTok{))) }\OperatorTok{+}
\StringTok{  }\KeywordTok{geom_sf_text}\NormalTok{(}\DataTypeTok{data =}\NormalTok{ temp, }\KeywordTok{aes}\NormalTok{(}\DataTypeTok{label=}\NormalTok{nombre), }\DataTypeTok{check_overlap =}\NormalTok{ T, }\DataTypeTok{size =} \FloatTok{1.5}\NormalTok{) }\OperatorTok{+}
\StringTok{  }\KeywordTok{theme_bw}\NormalTok{()}
\end{Highlighting}
\end{Shaded}

\includegraphics{proyecto_files/figure-latex/mapa-temp-1.pdf} \#\#\#
Importar DEM

Ahora necesitamos traer el DEM, que en este caso será uno resumido a
partir del SRTM-90m.

\begin{Shaded}
\begin{Highlighting}[]
\NormalTok{dem <-}\StringTok{ }\KeywordTok{read_stars}\NormalTok{(}\StringTok{'material-de-apoyo-master/data/dem_srtm_remuestreado.tif'}\NormalTok{)}
\KeywordTok{names}\NormalTok{(dem) <-}\StringTok{ 'ele'}
\KeywordTok{plot}\NormalTok{(dem)}
\end{Highlighting}
\end{Shaded}

\includegraphics[width=800px]{proyecto_files/figure-latex/dem-1} Ahora
remuestreamos el DEM para que se alinee con la cuadrícula fuente,
\texttt{grd}. El DEM remuestreado será la cuadrícula del covariable
(variable independiente) que utilizaremos para predecir el valor de
temperatura.

\begin{Shaded}
\begin{Highlighting}[]
\NormalTok{grdcovars <-}\StringTok{ }\KeywordTok{aggregate}\NormalTok{(dem, grd, mean, }\DataTypeTok{na.rm=}\NormalTok{T)}
\KeywordTok{plot}\NormalTok{(grdcovars)}
\end{Highlighting}
\end{Shaded}

\includegraphics[width=800px]{proyecto_files/figure-latex/remuestrear-dem-1}
\#\#\# Extraer datos de elevación y generar modelo

Necesitamos que los datos de elevación pasen al objeto \texttt{temp}, de
manera que podamos probar un modelo lineal que ponga en relación a la
elevación con la temperatura.

\begin{Shaded}
\begin{Highlighting}[]
\NormalTok{temp}\OperatorTok{$}\NormalTok{ele <-}\StringTok{ }\KeywordTok{st_as_sf}\NormalTok{(}\KeywordTok{aggregate}\NormalTok{(grdcovars, temp, mean))[[}\DecValTok{1}\NormalTok{]]}
\NormalTok{temp}\OperatorTok{$}\NormalTok{ele}
\end{Highlighting}
\end{Shaded}

\begin{verbatim}
##  [1]   37.04137  111.19428   96.23608   89.05470   53.63334  360.12991
##  [7]   49.84021  235.65909   83.14058  155.80394 1373.07323   57.56667
## [13]   82.44868  218.63500 1042.68780  122.73364  419.19694  151.41004
## [19]   94.38337   79.46646  115.23488 1355.50136  945.64083   66.49686
## [25] 1152.72653   29.14500  165.08992   24.01264   21.70497  443.20785
## [31]   41.29578  108.17977  167.63144   25.13954   66.00372   56.24511
## [37]   -4.08697   50.94419   10.78244  140.73771   41.89802 1010.59454
## [43]   11.03355  879.47538  617.60780   84.08008   14.71723  157.90341
## [49]   99.80911   58.09217  489.49667  647.02488   61.22240  212.92748
## [55]  191.10207   29.83365   24.18545  181.29521   79.46121  241.98926
## [61]  498.26058  529.88060   11.89785   53.39107   22.52430   26.70314
## [67]  237.12488   44.10347   72.15647  271.38220         NA  690.64947
\end{verbatim}

\begin{Shaded}
\begin{Highlighting}[]
\NormalTok{temp <-}\StringTok{ }\NormalTok{temp[}\OperatorTok{!}\KeywordTok{is.na}\NormalTok{(temp}\OperatorTok{$}\NormalTok{ele),] }\CommentTok{#Quitar observación con NA}
\KeywordTok{plot}\NormalTok{(temp}\OperatorTok{$}\NormalTok{tanual, temp}\OperatorTok{$}\NormalTok{ele)}
\end{Highlighting}
\end{Shaded}

\includegraphics[width=600px]{proyecto_files/figure-latex/agregar-y-modelo-1}

\begin{Shaded}
\begin{Highlighting}[]
\NormalTok{temp_lm <-}\StringTok{ }\KeywordTok{lm}\NormalTok{(tanual }\OperatorTok{~}\StringTok{ }\NormalTok{ele, temp)}
\KeywordTok{summary}\NormalTok{(temp_lm)}
\end{Highlighting}
\end{Shaded}

\begin{verbatim}
## 
## Call:
## lm(formula = tanual ~ ele, data = temp)
## 
## Residuals:
##     Min      1Q  Median      3Q     Max 
## -1.9474 -0.5766 -0.1059  0.6037  2.7525 
## 
## Coefficients:
##              Estimate Std. Error t value Pr(>|t|)    
## (Intercept) 26.724159   0.129657  206.11   <2e-16 ***
## ele         -0.004925   0.000314  -15.68   <2e-16 ***
## ---
## Signif. codes:  0 '***' 0.001 '**' 0.01 '*' 0.05 '.' 0.1 ' ' 1
## 
## Residual standard error: 0.8768 on 69 degrees of freedom
## Multiple R-squared:  0.7809, Adjusted R-squared:  0.7777 
## F-statistic: 245.9 on 1 and 69 DF,  p-value: < 2.2e-16
\end{verbatim}

\begin{Shaded}
\begin{Highlighting}[]
\KeywordTok{plot}\NormalTok{(temp_lm)}
\end{Highlighting}
\end{Shaded}

\includegraphics[width=600px]{proyecto_files/figure-latex/agregar-y-modelo-2}
\includegraphics[width=600px]{proyecto_files/figure-latex/agregar-y-modelo-3}
\includegraphics[width=600px]{proyecto_files/figure-latex/agregar-y-modelo-4}
\includegraphics[width=600px]{proyecto_files/figure-latex/agregar-y-modelo-5}

El modelo sugiere que existe asociación entre temperatura y elevación,
lo cual es esperable. En este caso, el gradiente resultante es de unos
-0.5°C por cada 100 metros de elevación. El gradiente comúnmente es de
-0.7°C/100m, pero en este caso, al utilizar un DEM resumido, el
gradiente igualmente se atenúa. Generemos variograma muestral con este
modelo.

\subsubsection{Variograma muestral}\label{variograma-muestral}

\begin{Shaded}
\begin{Highlighting}[]
\NormalTok{vt <-}\StringTok{ }\KeywordTok{variogram}\NormalTok{(tanual }\OperatorTok{~}\StringTok{ }\NormalTok{ele, temp)}
\NormalTok{vt}
\end{Highlighting}
\end{Shaded}

\begin{verbatim}
##     np       dist      gamma dir.hor dir.ver   id
## 1    1   5700.529 0.05627847       0       0 var1
## 2   32  16704.469 0.33635604       0       0 var1
## 3   48  24262.705 0.54374141       0       0 var1
## 4   71  33650.837 0.35356827       0       0 var1
## 5   95  43808.242 0.50038750       0       0 var1
## 6   96  53318.232 0.47861452       0       0 var1
## 7  119  63032.130 0.55137087       0       0 var1
## 8  114  72385.146 0.38117872       0       0 var1
## 9  117  82661.116 0.64274923       0       0 var1
## 10 132  91995.843 0.62593645       0       0 var1
## 11 121 101582.482 0.67431891       0       0 var1
## 12 133 111087.873 0.66429574       0       0 var1
## 13 127 120784.573 0.93949779       0       0 var1
## 14 137 130797.186 0.74052779       0       0 var1
## 15 119 140308.927 0.76607643       0       0 var1
\end{verbatim}

\begin{Shaded}
\begin{Highlighting}[]
\KeywordTok{plot}\NormalTok{(vt)}
\end{Highlighting}
\end{Shaded}

\includegraphics[width=800px]{proyecto_files/figure-latex/vgm-temp-1}
\#\#\# Variograma modelo

Parecería razonable utilizar un variograma modelo exponencial con rango
corto, por ejemplo, 20 o 30 km. Probemos.

\begin{Shaded}
\begin{Highlighting}[]
\NormalTok{vt_m <-}\StringTok{ }\KeywordTok{fit.variogram}\NormalTok{(vt, }\KeywordTok{vgm}\NormalTok{(}\DataTypeTok{model =} \StringTok{"Exp"}\NormalTok{, }\DataTypeTok{range =} \DecValTok{30000}\NormalTok{))}
\NormalTok{vt_m}
\end{Highlighting}
\end{Shaded}

\begin{verbatim}
##   model     psill   range
## 1   Exp 0.6121034 22069.9
\end{verbatim}

\begin{Shaded}
\begin{Highlighting}[]
\KeywordTok{plot}\NormalTok{(vt, vt_m, }\DataTypeTok{plot.numbers =}\NormalTok{ T)}
\end{Highlighting}
\end{Shaded}

\includegraphics[width=800px]{proyecto_files/figure-latex/vgm-temp-ajus-1}
\#\#\# Kriging universal

Finalmnente, ejecutamos el kriging.

\begin{Shaded}
\begin{Highlighting}[]
\NormalTok{k_u <-}\StringTok{ }\KeywordTok{krige}\NormalTok{(tanual }\OperatorTok{~}\StringTok{ }\NormalTok{ele, temp, }\KeywordTok{st_rasterize}\NormalTok{(}\KeywordTok{st_as_sf}\NormalTok{(grdcovars)), vt_m)}
\end{Highlighting}
\end{Shaded}

\begin{verbatim}
## [using universal kriging]
\end{verbatim}

Finalmente, lo representamos.

\begin{Shaded}
\begin{Highlighting}[]
\KeywordTok{ggplot}\NormalTok{() }\OperatorTok{+}
\StringTok{  }\KeywordTok{geom_stars}\NormalTok{(}\DataTypeTok{data =}\NormalTok{ k_u, }\KeywordTok{aes}\NormalTok{(}\DataTypeTok{fill =}\NormalTok{ var1.pred, }\DataTypeTok{x =}\NormalTok{ x, }\DataTypeTok{y =}\NormalTok{ y)) }\OperatorTok{+}\StringTok{ }
\StringTok{  }\KeywordTok{scale_fill_gradientn}\NormalTok{(}\DataTypeTok{colours =} \KeywordTok{rev}\NormalTok{(}\KeywordTok{brewer.pal}\NormalTok{(}\DecValTok{9}\NormalTok{, }\DataTypeTok{name =} \StringTok{'RdBu'}\NormalTok{))) }\OperatorTok{+}
\StringTok{  }\KeywordTok{geom_sf}\NormalTok{(}\DataTypeTok{data =} \KeywordTok{st_cast}\NormalTok{(prov, }\StringTok{"MULTILINESTRING"}\NormalTok{)) }\OperatorTok{+}
\StringTok{  }\KeywordTok{geom_sf}\NormalTok{(}\DataTypeTok{data =}\NormalTok{ pre1998) }\OperatorTok{+}
\StringTok{  }\KeywordTok{geom_sf_text}\NormalTok{(}\DataTypeTok{data =}\NormalTok{ prov, }\KeywordTok{aes}\NormalTok{(}\DataTypeTok{label=}\NormalTok{TOPONIMIA), }\DataTypeTok{check_overlap =}\NormalTok{ T, }\DataTypeTok{size =} \DecValTok{2}\NormalTok{) }\OperatorTok{+}
\StringTok{  }\KeywordTok{theme_bw}\NormalTok{()}
\end{Highlighting}
\end{Shaded}

\includegraphics{proyecto_files/figure-latex/krige-uk-1.pdf} \#\#\# Nota
final

Dado que en este caso existe una fuerte desviación de los datos respecto
a una distribución normal, aun usando transformación logarítmica, se
recomienda aplicar otro tipo de transformación. En este caso, luce mejor
emplear \emph{Tukey Ladder of Powers} (escalera de potencias de Tukey).
Usaremos la función \texttt{transformTukey} del paquete
\texttt{rcompanion}, que cargaremos a continuación.

\begin{Shaded}
\begin{Highlighting}[]
\KeywordTok{library}\NormalTok{(rcompanion)}
\NormalTok{temp}\OperatorTok{$}\NormalTok{tanualtrans <-}\StringTok{ }\KeywordTok{transformTukey}\NormalTok{(temp}\OperatorTok{$}\NormalTok{tanual, }\DataTypeTok{plotit =}\NormalTok{ F)}
\KeywordTok{hist}\NormalTok{(temp}\OperatorTok{$}\NormalTok{tanualtrans)}
\KeywordTok{qqnorm}\NormalTok{(temp}\OperatorTok{$}\NormalTok{tanualtrans)}
\KeywordTok{shapiro.test}\NormalTok{(temp}\OperatorTok{$}\NormalTok{tanualtrans)}
\end{Highlighting}
\end{Shaded}

\section{Resultados}\label{resultados}

Nótese que se trata de una transformación de potencia por medio del
exponente \texttt{8.1}. Para hacer la transformación inversa, bastaría
con aplicar \texttt{temp\$tanualtrans\^{}(1/8.1)}. Se podría ensayar con
la variable transformada para generar la superficie continua.

La metodología propuesta se aplicó en Republica Dominicana Para sus
diferntes provincias. logramos ejecutar un kriging ordinario para
predecir el valor de la precipitación de 1998 para todo el país a partir
de 21 observatorios. Notemos que se trataba de un kriging ordinario,
porque a la función \texttt{krige} no le introducimos una media
(argumento \texttt{beta}), e igualmente porque con la función
\texttt{variogram} generamos un variograma contra un intercepto (fórmula
\texttt{a1998log\textasciitilde{}1}).

realizamos un modelo lineal que pone la relación de la elevación con la
temperatura.

\ldots

\section{Discusión o Conclusiones}\label{discusiuxf3n-o-conclusiones}

concluimos que la precipitacion esta directamente relacionado con la
orografia del terreno y que existe asociación entre temperatura y
elevación. el gradiente resultante es de unos -0.5°C por cada 100 metros
de elevación. El gradiente comúnmente es de -0.7°C/100m.

utilizamos kriging universal para predecir el valor de la variable de
interés en función del modelo espacial aportado por el variograma Y, al
mismo tiempo, considerando covariables mediante polinomios.

\ldots

\section{Información de soporte}\label{informaciuxf3n-de-soporte}

Utilizamos codigos y procedimientos de la clase de datos puntuales y
GeoEstadistica del profesor Jose Martinez Batlle.

\ldots

\section{\texorpdfstring{\emph{Script}
reproducible}{Script reproducible}}\label{script-reproducible}

\ldots

\section{Referencias}\label{referencias}

Material de Apoyo suministrado por el profesor Jose Martinez Batlle.
Capa de division de provincias de la ONE (Oficina Nacional de
Estadisticas) Datos de Presipitacion Anual de La Oficina Nacional de
Meterologia (ONAMET) Capa ProvCenso2010 de ONE (Oficina Nacional de
Estadisticas) Datos de Temperatura(onamet\_temp\_anual) Registrada en
Observatorios de ONAMET. utilizamos el DEM - SRTM-90m
(dem\_srtm\_remuestreado.tif)




\newpage
\singlespacing 
\end{document}
